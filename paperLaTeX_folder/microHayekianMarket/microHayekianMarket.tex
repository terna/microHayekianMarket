\documentclass[12pt]{report}
\setcounter{tocdepth}{4}
\setcounter{secnumdepth}{4}

% to insert a title at the beginning of each page
% http://ctan.mirror.garr.it/mirrors/CTAN/info/italian/fancyhdr/itfancyhdr.pdf
\usepackage{fancyhdr}
\pagestyle{fancy}
\lhead{}
\rhead{}
\chead{micro Hayekian Marketl}


\usepackage{geometry}                % See geometry.pdf to learn the layout options. There are lots.
\geometry{a4paper}                   % ... or a4paper or a5paper or ... letterpaper
%\geometry{landscape}                % Activate for for rotated page geometry
%\usepackage[parfill]{parskip}    % Activate to begin paragraphs with an empty line rather than an indent
\usepackage{graphicx}
\usepackage{amssymb}
\usepackage{epstopdf}
\usepackage[hyphens]{url}         %[hyphens] to break long urls
\usepackage{t1enc} %per uso di caratteri come << >>
%\DeclareGraphicsRule{.tif}{png}{.png}{`convert #1 `dirname #1`/`basename #1 .tif`.png}

\usepackage{verbatim} %to use \begin{comment} \end{comment}
\usepackage{float} % to use H as rigid placement of a figure after the text, allowing empty spaces

\usepackage{fancyvrb}  % http://texdoc.net/texmf-dist/doc/latex/fancyvrb/fancyvrb.pdf

\usepackage{enumitem} % to avoid bold in description item and to maintain bullets

\usepackage[toc,page]{appendix}

\usepackage{keystroke} % to use \Return

\newcommand{\repeatfootnote}[1]{\textsuperscript{\ref{#1}}} %to repeat a reference


% to have clickable links and references
\usepackage[dvipsnames]{xcolor} % colors at https://en.wikibooks.org/wiki/LaTeX/Colors
\usepackage{xcolor}   %Maybe necessary if you want to color links (better use xcolor than color, more at http://repositorios.cpai.unb.br/ctan/macros/latex/contrib/xcolor/xcolor.pdf)
\usepackage{hyperref}
\hypersetup{
    colorlinks=true, %set true if you want colored links
    linktoc=all,     %set to all if you want both sections and subsections linked
    linkcolor=Brown,  %choose some color if you want links to stan
    urlcolor=cyan,
    citecolor=purple}
    
\usepackage{chngcntr}                      % to avoid restart note numbers 
\counterwithout{footnote}{chapter}    % changing chapter

\usepackage{makeidx}   % creating index (in Italian Indice analitico)

\usepackage{empheq}

\usepackage{tablefootnote} % to use \footnote as \tablefootnote

\newcommand{\ts}{\textsuperscript} %to write briefly th etc. as a superscript

\usepackage{color}
\usepackage{listings}


\makeindex



\title{micro Hayekian Market}
%\subtitle{a}
\author{Matteo Morini\footnote{University of Torino, Italia} and Pietro Terna\footnote{University of Torino, Italia}}

%\date{} %attivare vuoto per eliminare la data oppure attivarne una dichiarata

%\usepackage[square]{natbib}
\usepackage[round]{natbib}

\setlength\fboxsep{0pt}

\renewcommand{\thesection}{\arabic{section}} % to avoid leading zeros in numbering the sections, having eliminate
                                                                           % the Chapter 1 supertitle (see below: to eliminate the Chapter 1 supertitle)

\begin{document}
\maketitle
\thispagestyle{fancy}

\tableofcontents
\thispagestyle{fancy}

\listoffigures
\thispagestyle{fancy}


%%%%%%%%%%%%%%%%%%%%%%%%%%%%%%%%%%%%%%%%%%%%
%%%%%%%%%%%%%%%%%%%%%%%%%%%%%%%%%%%%%%%%%%%%
\chapter*{Introduction to a micro Hayekian Market}
%The * above is to eliminate the Chapter 1 supertitle
\label{micro Hayekian Market}\index{introduction to a micro Hayekian Market}
\thispagestyle{fancy}
\addcontentsline{toc}{chapter}{\protect\numberline{}Introduction to a micro Hayekian Market}%


The purpose of the note is that of introducing a very simple agent-based model of a market, with emergent (quite interesting)  price dynamics.

 A counter example is also introduced, showing how with tiny modification we generate implausible price dynamics.

The code uses the IPython\footnote{\url{https://ipython.org}.} language (interaction with Python\footnote{\url{https://www.python.org}.}) and can be dowloaded from \url{https://github.com/terna/microHayekianMarket} using the \emph{Clone or download} button; it is also possible to run it directly on line at \\\url{https://mybinder.org/v2/gh/terna/microHayekianMarket/master?filepath=microHayekianMarket.ipynb}.

A suggested reading about Hayek is a quite recent paper of \citeauthor{10.1257/jep.31.3.215} (\citeyear{10.1257/jep.31.3.215}).


%%%%%%%%%%%%%%%%%%%%%%%%%%%%%%%%%%%%%%%%%%%%
\section{The technical setup}\label{The technical setup}\index{technical setup}

The IPython (or Python 3.x) code requires the following starting setup:

\begin{lstlisting}[language=Python, caption=Warming up of the model, basicstyle=\ttfamily\footnotesize]
%pylab inline
import statistics as s
import numpy as np
import pylab as plt
from IPython.display import clear_output
import time
\end{lstlisting}

\verb|%pylab inline| 
is a \emph{magic} command of Jupyter.\footnote{\url{http://jupyter.org}.}

%%%%%%%%%%%%%%%%%%%%%%%%%%%%%%%%%%%%%%%%%%%%
\section{The structure of the model and the \emph{warming up} phase}\label{The structure of the model}\index{structure}

Our agents are simply prices, to be interpreted as reservation prices.\footnote{The $max$ price a buyer could pay and the $min$ one a seller could accept.}

We have two price vectors: $pL^b$ with item $pL^b_i$ for the buyers, and $pL^s$ with item $pL^s_j$ for the sellers. The $i^{th}$ or the $j^{th}$ elements of the vectors are prices, but in this case we can use them also as agents.

Both in the hayekian perspective (Section \ref{The hayekian version}) and in the unstructured one (Section \ref{The unstructured version}) we have a common \emph{warning up} action.

In this phase we define:

\begin{itemize}
\item $nCycles$ - number of simulation cycles;
\item $nBuyers$  - number of the buyers;
\item $nSellers$ - number of the sellers;
\item $d_0$ - the lower bound for random uniform numbers, both for the buyers and the sellers in the warming up phase; in the running phase, the lower bound is $0$;
\item $d_1$ - the upper bound for random uniform numbers for the buyers;
\item $d_2$ - the upper bound for random uniform numbers for the sellers;
\item the initial buyer $i$ reservation price, different for each buyer: $p_{b,i}=\frac{1} {1 + u_i}$ with $u_i\sim\mathcal{U}(d_0,d_1)$;
\item initial seller $j$ reservation price, different for each seller: $p_{s,j}=1 + u_j$ with $u_j\sim\mathcal{U}(d_0,d_2)$.
\end{itemize}

With  $d_0=0.1$, $d_1=0.2$, $d_2=0.2$ and sorting in decreasing order the vector  $pL^b$ and in increasing order the vector  $pL^s$ we obtain two not overlapping price sequences that we can interpret as a demand and an offer curves (Fig. \ref{output_2_1.png}).

\begin{figure}[htbp]
\begin{center}
\fbox{\centering \includegraphics[width=0.6\textwidth]{output_2_1.png}}
\caption{An example of initial not overlapping demand and offer curves}
\label{output_2_1.png}
\end{center}
\end{figure}

This is the \emph{warming up}, or starting situation, of the model. To generate new examples related to Section \ref{The hayekian version} and to Section \ref{The unstructured version}, it is necessary to repeat this phase.

The IPython (or Python 3.x) code is:

\begin{lstlisting}[language=Python, caption=Warming up of the model, basicstyle=\ttfamily\footnotesize]
# warming up

# run it before executing both - the hayekian perspective or
#                              - the unstructured case

d0=0.1
d1=0.2
d2=0.2

nCycles=10000
nBuyers= 100
nSellers=100

buyerPriceList=[]
sellerPriceList=[]

for i in range(nBuyers):
    buyerPriceList.append(1/(1+np.random.uniform(d0,d1)))
for j in range(nSellers):
    sellerPriceList.append(1+np.random.uniform(d0,d2))
    
plt.plot(np.sort(buyerPriceList)[::-1],"r");
plt.plot(np.sort(sellerPriceList),"b");
xlabel("the agents")
ylabel("agents' reservation prices")
\end{lstlisting}

%%%%%%%%%%%%%%%%%%%%%%%%%%%%%%%%%%%%%%%%%%%%
\section{The hayekian version}\label{The hayekian version}\index{hayekian version}
 
The buyers and the sellers meet randomly. Buyer $i$ and seller $j$ exchange if  $pL^b_i \geq pL^s_j$; the deal is recorded at the price of the seller $pL^s_j$.\footnote{In the $mall$ sell price are public.}

In this version, which is the important one in this note, the running prices are changing following the correction coefficients:

\begin{itemize}
\item for the buyer: (i) $c_b=\frac{1} {1 + u_b}$ if the deal succeeds (trying the pay less next time) or (ii) $c_b=1 + u_b$ if the deal fails (preparing to pay more next time); in (i) and (ii) we have $u_b\sim\mathcal{U}(0,d_1)$

\item for the seller: (iii) $c_s=\frac{1} {1 + u_s}$ if the deal succeeds (preparing to obtain a higher revenue next time) or (iv) $c_s=1 + u_s$ if the deal fails (preparing to obtain a lower revenue next time); in (iii) and (iv) we have $u_s\sim\mathcal{U}(0,d_2)$.
\end{itemize}

With $d_1=0.2$, $d_2=0.2$ and $nCycles$ set to $10,000$ we obtain sequences of mean prices (mean in each cycle) quite realistic, with a very low variance within each cycle (see Fig. \ref{output_3_1.png} and \ref{output_3_2.png}).

\begin{figure}[htbp]
\begin{center}
\fbox{\centering \includegraphics[width=0.6\textwidth]{output_3_1.png}}
\caption{Hayekian case: (i) an example of final demand and offer curves, (ii) the history of mean prices, (iii) their coeffcients of variation within each cycle}
\label{output_3_1.png}
\end{center}
\end{figure}

The \emph{coefficient of variation} is calculated as $\frac{standard~deviation}{mean}$.

\begin{figure}[htbp]
\begin{center}
\fbox{\centering \includegraphics[width=0.6\textwidth]{output_3_2.png}}
\caption{Hayekian case: (i) distribution of mean prices in each cycle and (ii) of their standard deviations}
\label{output_3_2.png}
\end{center}
\end{figure}

A comment: we have a plausible series of mean price, with a complicated behavior, and with a high stability of the dispersion of the values within each cycle.

The right side of the buyer and seller curves shows another plausible situation with agents not exchanging. 

The IPython (or Python 3.x) code is:

\begin{lstlisting}[language=Python, caption=Warming up of the model, basicstyle=\ttfamily\footnotesize]
# hayekian perspective
meanPrice_ts=[]
meanPriceStDev_ts=[]
meanPriceVar_ts=[]

for t in range(1,nCycles+1):    
    dealPrices=[]
    agNum=max(nBuyers,nSellers)
    for n in range(agNum):
        i = np.random.randint(0,nBuyers)
        j = np.random.randint(0,nSellers)
        #print ('%2d %2d %.3f %.3f %.3f'% \
        #      (i,j,buyerPriceList[i]-sellerPriceList[j],\
        #       buyerPriceList[i],sellerPriceList[j]))
        
        if buyerPriceList[i]>=sellerPriceList[j]:
            dealPrices.append(sellerPriceList[j])
            buyerPriceList[i] *=1/(1+np.random.uniform(d1))
            sellerPriceList[j]*=1+np.random.uniform(d2)
        else:
            buyerPriceList[i] *=1+np.random.uniform(d1)
            sellerPriceList[j]*=1/(1+np.random.uniform(d2))

        #print ('%2d %2d %.3f %.3f %.3f \n'% \
        #      (i,j,buyerPriceList[i]-sellerPriceList[j],\
        #       buyerPriceList[i],sellerPriceList[j]))
           
    if len(dealPrices) > 2:
        meanPrice_ts.append(s.mean(dealPrices))
        meanPriceVar_ts.append(s.variance(dealPrices))
        meanPriceStDev_ts.append(s.stdev(dealPrices))
    else:
        meanPrice_ts.append(np.nan)
        meanPriceStDev_ts.append(np.nan)

    if t % 1000==0:
        clear_output()
        print('time', t, 'and n. of exchanges in the last cycle', \
              len(dealPrices))
        print(\
        'mean and var of exchange prices in the last cycle: %.3f, %.3f' %\
              (meanPrice_ts[-1],meanPriceVar_ts[-1]))

        plt.figure(1,figsize=(7,15),clear=True)

        plt.subplot(311)
        plt.plot(np.sort(buyerPriceList)[::-1],"r")
        plt.plot(np.sort(sellerPriceList),"b")
        plt.title(\
            "buyers' prices (red) and sellers' prices (blue)")
        xlabel("the agents")
        ylabel("agents' reservation prices")

        plt.subplot(312)
        plt.title("mean price of each cycle")
        xlabel("t")
        ylabel("mean price of each cycle")
        plt.plot(meanPrice_ts,"g")
        
        plt.subplot(313)
        plt.title("price coef. of variation within each cycle")
        coefOfVariation=[]
        for m in range(len(meanPriceStDev_ts)):
            coefOfVariation.append(meanPriceStDev_ts[m]/
                                   meanPrice_ts[m])
        plt.plot(coefOfVariation,".",markersize=0.1)
        xlabel("t")
        ylabel("price coef. of variation within each cycle")
        show()
        #time.sleep(0.1)

plt.figure(2,figsize=(7,9))
plt.subplot(211)
plt.title("mean price histogram")
plt.hist(meanPrice_ts,100,color="g");
plt.subplot(212)
plt.title("price standard deviation (within each cycle) histogram")
plt.hist(meanPriceStDev_ts,100);
\end{lstlisting}


%%%%%%%%%%%%%%%%%%%%%%%%%%%%%%%%%%%%%%%%%%%%
\section{The unstructured version}\label{The unstructured version}\index{unstructured version}

The buyers and the sellers meet randomly as in Section \ref{The hayekian version}. Buyer $i$ and seller $j$ exchange in any case; the deal is recorded at the mean of the price of the seller $pL^s_j$ and of the price of the buyer.

In this version the running prices are changing following the correction coefficients:

\begin{itemize}

\item with equal probability for the buyer: (i) $c_b=\frac{1} {1 + u_b}$ or (ii) $c_b=1 + u_b$); in (i) and (ii) we have $u_b\sim\mathcal{U}(0,d_1)$

\item  with equal probability for the seller: (iii) $c_s=\frac{1} {1 + u_s}$ or (iv) $c_s=1 + u_s$; in (iii) and (iv) we have $u_s\sim\mathcal{U}(0,d_2)$.
\end{itemize}

With $d_1=0.2$, $d_2=0.2$ and $nCycles$ set to $10,000$ we obtain exploding sequences of mean prices (mean in each cycle), and explodinf variance within each cycle (see Fig. \ref{output_4_1.png} and \ref{output_4_2.png}).

\begin{figure}[htbp]
\begin{center}
\fbox{\centering \includegraphics[width=0.6\textwidth]{output_4_1.png}}
\caption{Unstructured case: (i) an example of final demand and offer curves, (ii) the history of mean prices, (iii) their coeffcients of variation within each cycle}
\label{output_4_1.png}
\end{center}
\end{figure}

The \emph{coefficient of variation} is calculated as $\frac{standard~deviation}{mean}$.

\begin{figure}[htbp]
\begin{center}
\fbox{\centering \includegraphics[width=0.6\textwidth]{output_4_2.png}}
\caption{Unstructured case: (i) distribution of mean prices in each cycle and (ii) of their standard deviations}
\label{output_4_2.png}
\end{center}
\end{figure}

A comment: this counter-example shows the missing the intelligent correction of the price that implicitly propagate the price among all the agents, a system of pure random price settings is absolutely far from being plausible.

The IPython (or Python 3.x) code is:

\begin{lstlisting}[language=Python, caption=Warming up of the model, basicstyle=\ttfamily\footnotesize]
# unstructured case (remember the warming up step)
meanPrice_ts=[]
meanPriceStDev_ts=[]
meanPriceVar_ts=[]

for t in range(1,nCycles+1):    
    dealPrices=[]
    agNum=max(nBuyers,nSellers)
    for n in range(agNum):
        i = np.random.randint(0,nBuyers)
        j = np.random.randint(0,nSellers)
        #print ('%2d %2d %.3f %.3f %.3f'% \
        #      (i,j,buyerPriceList[i]-sellerPriceList[j],\
        #       buyerPriceList[i],sellerPriceList[j]))
        
        dealPrices.append((sellerPriceList[j]+buyerPriceList[i]/0.5))
        
        if np.random.uniform(0,1)>=0.5:    
            buyerPriceList[i] *=1/(1+np.random.uniform(0,d1))
            sellerPriceList[j]*=1+np.random.uniform(0,d2)
        else:
            buyerPriceList[i] *=1+np.random.uniform(0,d1)
            sellerPriceList[j]*=1/(1+np.random.uniform(0,d2))

        #print ('%2d %2d %.3f %.3f %.3f \n'% \
        #      (i,j,buyerPriceList[i]-sellerPriceList[j],\
        #       buyerPriceList[i],sellerPriceList[j]))
           
    if len(dealPrices) > 2:
        meanPrice_ts.append(s.mean(dealPrices))
        meanPriceVar_ts.append(s.variance(dealPrices))
        meanPriceStDev_ts.append(s.stdev(dealPrices))
    else:
        meanPrice_ts.append(np.nan)
        meanPriceStDev_ts.append(np.nan)

    if t % 1000==0:
        clear_output()
        print('time', t, 'and n. of exchanges in the last cycle', \
              len(dealPrices))
        print(\
        'mean and var of exchange prices in the last cycle: %.3f, %.3f' %\
              (meanPrice_ts[-1],meanPriceVar_ts[-1]))

        plt.figure(1,figsize=(7,15),clear=True)

        plt.subplot(311)
        plt.plot(np.sort(buyerPriceList)[::-1],"r")
        plt.plot(np.sort(sellerPriceList),"b")
        plt.title(\
            "buyers' prices (red) and sellers' prices (blue)")
        xlabel("the agents")
        ylabel("agents' reservation prices")

        plt.subplot(312)
        plt.title("mean price of each cycle")
        xlabel("t")
        ylabel("mean price of each cycle")
        plt.plot(meanPrice_ts,"g")
        
        plt.subplot(313)
        plt.title("price coef. of variation within each cycle")
        coefOfVariation=[]
        for m in range(len(meanPriceStDev_ts)):
            coefOfVariation.append(meanPriceStDev_ts[m]/
                                   meanPrice_ts[m])
        plt.plot(coefOfVariation,".",markersize=0.1)
        xlabel("t")
        ylabel("price coef. of variation within each cycle")
        show()
        #time.sleep(0.1)

plt.figure(2,figsize=(7,9))
plt.subplot(211)
plt.title("mean price histogram")
plt.hist(meanPrice_ts,100,color="g");
plt.subplot(212)
plt.title("price standard deviation (within each cycle) histogram")
plt.hist(meanPriceStDev_ts,100);
\end{lstlisting}


\clearpage
\addcontentsline{toc}{chapter}{Bibliography}
\bibliography{./bibliografiaGenerale}
\bibliographystyle{plainnatmm}


\clearpage
\addcontentsline{toc}{chapter}{Index}
\printindex


\end{document}  





